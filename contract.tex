\documentclass[]{article}
\newcommand{\mytitle}{SENIOR CARE CONTRACT}
\newcommand{\agreementtitle}{WILSON MANOR, BEDROOM IN CONDOMINIUM APARTMENT 13, WEEK-TO-WEEK LODGING RENTAL AGREEMENT}
\newcommand{\rulestitle}{SENIOR CARE GUIDELINES}
\newcommand{\healthtitle}{HEALTH RECORDS}
\overfullrule=2cm
\usepackage[margin=1in,headheight=28pt]{geometry}
\usepackage[table]{xcolor}
\definecolor{lightgray}{gray}{0.9}
\definecolor{medgray}{gray}{0.8}
\newcommand{\checkbox}{\raisebox{2pt}{\framebox[12pt][c]{\rule[7pt]{0pt}{-.3\baselineskip}}}}

\newcommand{\startdate}{the/start/date}
\newcommand{\datefillin}{\hspace{0.2cm}\rule{3cm}{.1pt}}
\newcommand{\initialfillin}{\hspace{0.2cm}\rule{1.5cm}{.1pt}}
\newcommand{\sw}{.15\textwidth}
\newcommand{\nw}{.2\textwidth}
\newcommand{\mw}{.29\textwidth}
\newcommand{\bw}{.39\textwidth}
\newcommand{\tabtiny}[1]{\makebox[1.8em][l]{#1}\ignorespaces}
\newcommand{\tabsmall}[1]{\makebox[.14\linewidth][l]{#1}\ignorespaces}
\newcommand{\tabmed}[1]{\makebox[.22\linewidth][l]{#1}\ignorespaces}
\newcommand{\tablmed}[1]{\makebox[.33\linewidth][l]{#1}\ignorespaces}
\newcommand{\tablarge}[1]{\makebox[.55\linewidth][l]{#1}\ignorespaces}
\newcommand{\lname}{Caregiver's Name}
\newcommand{\mom}{Joan B. Harrison}
\newcommand{\allweek}{ALL-WEEK non-LIVE-IN Employee}
\newcommand{\weekend}{WEEKEND non-LIVE-IN Employee}
\newcommand{\weekday}{WEEKDAY LIVE-IN Employee}

\input{attachmenttitles}

\newcounter{attachmentcounter}
\renewcommand{\theattachmentcounter}{\Alph{attachmentcounter}}
\newenvironment{attachment}[1] {%
	\refstepcounter{attachmentcounter}%
	\noindent \textbf{\Large Attachment\theattachmentcounter:~{#1}}
	\noindent
}{}

\newcommand{\makecell}[2][c]{%
  \begin{tabular}[#1]{@{}l@{}}#2\end{tabular}}

\usepackage{multicol}
\usepackage{fancyhdr}
\pagestyle{fancy}
\fancypagestyle{withfooter}{
	\fancyhf{}
	\lhead{\centering \mytitle{}}
	\rfoot{Page \thepage~of~\pageref{LastPage}}
}
\fancypagestyle{rules}{
	\fancyhf{}
	\lhead{\centering \rulestitle{}}
	\rfoot{Page \thepage~of~\pageref{LastRulesPage}}
}
\fancypagestyle{health}{
	\fancyhf{}
	\lhead{\centering \healthtitle{}}
	\rfoot{Page \thepage~of~\pageref{LastHealthPage}}
}
\fancypagestyle{empty}{
	\fancyhf{}
}
\renewcommand{\thesection}{\Roman{section}}

\pagestyle{withfooter}
\begin{document}
\title{\mytitle{}}
\author{Drafted by Colin N Keenan colinnkeenan@gmail.com}
\date{April 2015}
\maketitle
\thispagestyle{fancy}

\noindent \hrulefill
\section*{Terms of Employment}
Dear \lname{},\\

Thank you for providing this much-needed care! The following contract is to make sure we are all on the same page about responsibilities, sick days, taxes, payments and schedules. The attached ``\rules{}'' (Attachment \ref{rules}), ``\basic{}'' (Attachment \ref{basic}), ``\bathroom{}'' (Attachment \ref{bathroom}), ``\shopping{}'' (Attachment \ref{shopping}), ``\bigchores{}'' (Attachment \ref{bigchores}), and ``\health{}'' (Attachment \ref{health}) offer a little more information about how our family works and how we are hoping you can assist our loved one, written down so it can be clear. While the below contract is very black and white, the attachments will be agile documents as we know things might change.

This contract, executed on \datefillin{}, between \mom{} (``Employer'') and \lname{} (``Employee'') has the following terms of employment:

\subsection*{Start Date}

Employee will start employment on \textbf{\startdate{}} and continue until either party elects to terminate the relationship.

\subsection*{Worksite Address}

Work will be performed at 104 North Wilson Ave, Apt. 13, Pasadena, CA 91106 (``Employer's Home'').

The Employee may choose to rent a room in the Employer's Home upon the terms set forth in the \agreementtitle{}.

\subsection*{Notes About The Person Requiring Care}

\input{impairmentcontent}

She is able to ascend and descend stairs, use the bathroom alone and is able to feed herself.

The person you will care for can be left alone.

\subsection*{Work Schedule, Responsibilities, and Compensation}
The work schedule, responsibilities, and compensation depend on whether or not the Employee is renting a room in the Employer's Home, and how many days a week the Employee is allowed to work.
\subsubsection*{Three Categories of Employee}

\noindent Employee is a category\\
\noindent(check one)
\begin{itemize}
	\item[\checkbox] \ref{allweek}. \allweek{}
	\item[\checkbox] \ref{weekend}. \weekend{}
	\item[\checkbox] \ref{weekday}. \weekday{}
\end{itemize}

Regardless of what category of employee your are, you will need to check-in by phone at the start of each shift and check-out by phone or internet at the end of each shift as set forth in the \rules{}. Your daily morning and evening chores are set forth in \basic{}, \bathroom{}, and \shopping{}. 

You will \textbf{not} be responsible for doing anything listed in ``\notyourjob{}'' (Attachment \ref{notyourjob}). 

Further responsibilities, your work schedule, and compensation are listed under your category of employee below.
\begin{enumerate}
	\item \textbf{\allweek{}}. \label{allweek}
		Is not renting a room in the Employer's Home. Is required to work six (6) days a week, and be available for an occasional 7th day of work. For any 7th day of work done in a calendar month, the employee will have another day off in another week that month, and so will work less than six (6) days sometimes. In addition to the responsibilities already listed, once a week, you will do a bigger chore from the \bigchores{} list. You will also assist with showers a few times a month as set forth in \shower{}. 
		\begin{enumerate}
			\item \textbf{Morning Shift}. 8am until chores are completed or Employee is out of time (typically finished at about 9am, but may finish anytime between 8:30am and 10am)
			\item \textbf{Evening Shift}. 7pm until chores are completed or Employee is out of time (typically finished at about 8pm, but may finish anytime between 7:30pm and 9pm)
			\item \textbf{14 Average Weekly Hours Plus Medical Appointments}. The average weekly hours in any calendar month, not counting medical and dental appointments, should be within half an hour of fourteen (14). 
				\begin{enumerate}
					\item Since 7-day work-weeks must be balanced by shorter ones, the actual hours worked each week depends on the number of days worked:\\
						\rowcolors{1}{lightgray}{white}
						\noindent\begin{tabular}{|p{\sw}|p{\sw}|}
							\hline
							\rowcolor{medgray}
							Days Worked & Hours Allowed (Not Counting Appointments) \\ \hline
							7 & 15.5 - 16.5 \\ \hline
							6 & 13.5 - 14.5 \\ \hline
							5 & 11.5 - 12.5 \\ \hline
							4 & 9.5 - 10.5 \\ \hline
							3 & 7.5 - 8.5 \\
							\hline
						\end{tabular}
					\item The Employee must be able to make two weekdays available each week for accompanying Employer to medical and dental appointments as scheduled by Employee.
					\item The Employee shall attempt to schedule no more than two medical appointments in the same work-week.
					\item The Employee may leave certain chores to be completed on the following shift, or do some chores in advance that would normally be done on the following shift. As long as medicine, insulin, and food is served on time, all other chores can be done on the previous or following shift.
				\end{enumerate}
		\end{enumerate}
		\begin{tabular}{|p{\bw}|p{\sw}|p{\sw}|p{\sw}|}
			\hline
			\rowcolor{medgray}
			Pay Rate Type & Amount & x Estimated Hours per week & = Estimated Weekly Totals\\ \hline
			Regular rate of pay & \$13.26 per hour & 14 hours & \$185.64\\ \hline
			Medical or Dental Appointments (varies weekly, but no more than 2 appointments per week) & \$13.26 per hour & 6 hours & \$79.56\\ \hline
			Overtime rate of pay when working 7 days in the same work-week (1.5 x the regular rate of pay) for any hours worked over thirty (30) in a week or over six (6) in a day & \$19.89 per hour & 0 hours & \$0.00\\ \hline
			Weekly Totals: & &  20 hours & \$265.20\\
			\hline
		\end{tabular}
	\item \textbf{\weekend{}}. \label{weekend}
		Is not renting a room in the Employer's Home. Is required to work on Saturday and/or Sunday, but not on weekdays. May work additional days when the weekday employee is not available.
		\begin{enumerate}
			\item \textbf{Morning Shift}. 8am - 9am
			\item \textbf{Evening Shift}. 7pm - 8pm 
			\item \textbf{4 Weekly Hours}. The total number of hours will be no more than four (4) each week, unless working another employee's shift. There will not be any medical or dental appointments. 
		\end{enumerate}
		\begin{tabular}{|p{\bw}|p{\sw}|p{\sw}|p{\sw}|}
			\hline
			\rowcolor{medgray}
			Pay Rates & Amounts & x Estimated Hours per week & Estimated Weekly Totals\\ \hline
			Regular rate of pay & \$13.26 per hour & 2 - 4 hours & \$26.52 - \$53.04\\ 
			\hline
		\end{tabular}
	\item \textbf{\weekday{}}. \label{weekday}
		Is renting a room in the Employer's Home. Is required to work five (5) days a week, but not on weekends. May trade days with the weekend employee as long as the total days worked are 5 or less. Would only work on the 6th or 7th day in an emergency (which is an unpredictable or unavoidable occurrence at unscheduled intervals requiring immediate action). In addition to the responsibilities already listed, once a week, you will do a bigger chore from the \bigchores{} list. You will also assist with showers a few times a month as set forth in \shower{}. 
		\begin{enumerate}
			\item \textbf{Morning Shift}. 8am until chores are completed or Employee is out of time (typically finished at about 9am, but may finish anytime between 8:30am and 10am)
			\item \textbf{Evening Shift}. 7pm until chores are completed or Employee is out of time (typically finished at about 8pm, but may finish anytime between 7:30pm and 9pm)
			\item \textbf{12 - 14 Weekly Hours Plus Medical Appointments}. The total number of hours, not counting medical and dental appointments, should be within an hour of thirteen (13) hours each week. 
				\begin{enumerate}
					\item The Employee must be able to make two weekdays available each week for accompanying Employer to medical and dental appointments as scheduled by Employee.
					\item The Employee shall attempt to schedule no more than two appointments in the same work week.
					\item The Employee may leave certain chores to be completed on the following shift, or do some chores in advance that would normally be done on the following shift. As long as medicine, insulin, and food is served on time, all other chores can be done on the previous or following shift.
				\end{enumerate}
		\end{enumerate}
		\begin{tabular}{|p{\bw}|p{\sw}|p{\sw}|p{\sw}|}
			\hline
			\rowcolor{medgray}
			Pay Rates & Amounts & x Estimated Hours per week & Estimated Weekly Totals\\ \hline
			Regular rate of pay & \$13.26 per hour & 13 hours & \$172.38\\ \hline
			Medical or Dental Appointments (varies weekly, but no more than 2 appointments per week) & \$13.26 per hour & approx. 6 hours & \$79.56\\ \hline
			Time and a half (1.5 x the regular rate of pay) for any hours up to and including nine hours on the sixth (6th) and seventh (7th) workdays; for any hours worked over nine (9) in a day & \$19.89 per hour & 0 hours & \$0.00\\ \hline
			Double time (2 x the regular rate of pay) for hours worked over nine (9) hours on the sixth and seventh consecutive day of workweek. & \$26.52 per hour & 0 hours & \$0.00\\ \hline
			Weekly Totals: & & 19 hours & \$251.94\\
			\hline
		\end{tabular}
\end{enumerate}

\subsection*{Paychecks}
\begin{enumerate}
	\item{\textbf{Payment of Wages}}. Weekly (Every Monday). Wages will be paid directly into the Employee's Venmo account by no later than Noon each Monday for the work week that ended the day before. The work week will begin on Monday and will end on Sunday. Wages shall not be paid in cash.
	\item{\textbf{Payroll Log}}. 
		\begin{enumerate}
			\item The Employee will be required to check-in and check-out of each shift by phone or internet as set forth in \rules{}. 
			\item By 7am Monday, the Employer will email to Employee a daily log setting forth expenses and the total hours worked, along with check-in/check-out times, hours worked for each shift, and mileage during the work-week that ended at 11:59pm the night before (``Payroll Log'').
			\item By 9am the same Monday the Payroll Log was emailed, the Employee will send an email reply indicating either that the Payroll Log is accepted by, or that it is disputed by, the Employee. If disputed, the reply email should include an explanation and suggest additional, changed, or removed, check-in and check-out times or expenses.
			\item If the Employee's email reply to the Payroll Log has not been received by 8am, the Employer will attempt to contact the Employee by phone.
			\item If the Employee's email reply to the Payroll Log has not been received by 9am, the Employer will do payroll based on the Payroll Log, but the Employee is still required to email a reply to that Payroll Log.
		\end{enumerate}
	\item{\textbf{Mileage And General Expenses}}. Any miles driven while on the job using the employee's car will be reimbursed at the IRS Mileage Reimbursement Rate, which covers the cost of gasoline as well as general wear and tear on the car. Employee will maintain a mileage log and submit to Employer for reimbursement at the end of the pay period. The 2015 IRS mileage reimbursement rate is 57.5 cents per mile.  All other pre-approved, work-related expenses will be reimbursed at cost. Employee will keep all receipts and submit to employer for reimbursement at the end of each pay period.
	\item{\textbf{Paid Time Off}}. Employee will receive the following paid time off:
		\begin{itemize}
			\item Family Sick Leave (24 hours per year) will be accrued as set forth below. At least one (1) week's notice is requested for any appointments, etc. which may cause the Employee to miss work.
		\end{itemize} The Employee will qualify for paid sick leave by working for the Employer for at least 30 days within a year in California and by satisfying a 90 day employment period. The Employee will begin to accrue Family Sick Leave upon the first day of his or her employment at the rate of one hour paid leave for every 30 hours worked. However, the Employee may only take 24 hours of sick leave per year. Accrued paid sick leave may be carried over to the next year, but accrued sick leave may not exceed 48 hours.\\
	An employee can take paid leave for employee's own or a family member for the diagnosis, care or treatment of an existing health condition or preventive care or for specified purposes for an employee who is a victim of domestic violence, sexual assault or stalking or as otherwise permitted under California law. (``Family Sick Leave'').
	\item{\textbf{Tax Withholding/Reporting}}. Employee will complete Form I-9 (available at www.uscis.gov/forms) and provide the required documentation verifying employment eligibility within three days of hiring. Employer will withhold the required Social Security and Medicare taxes from the employee's pay, along with income taxes per the Employee's instructions on Form W-4 and all other applicable state taxes. All tax withholdings will be remitted to the state and federal tax agencies on or before the household employment tax deadlines. In addition, Employer will match the employee's Social Security and Medicare contributions and make contributions to the state and federal unemployment insurance funds on behalf of the employee. Employer will provide employee with Form W-2 at the end of the year (by January 31). Employer will report employee's earnings to the Social Security Administration so that employee receives appropriate retirement benefits.
	\item{\textbf{Raises}}. Cost Of Living Adjustment (``COLA''). After 5 years of employment, the Employee will be eligible for a percent raise equal to the cumulative United States Social Security COLA for those five years. This raise will be automatic.
\end{enumerate}

\subsection*{Confidentiality, At Will, and Social Media}
\begin{enumerate}
	\item{\textbf{Confidentiality}}. Employee understands that any and all private information obtained about the Employer during the course of employment, including but not limited to medical, financial, legal and career information, is strictly confidential and may not be disclosed to any third party for any reason.  
	\item{\textbf{At Will Employment}}. It is understood and agreed that:
		\begin{enumerate}
			\item The Employee may at any time terminate this agreement and his or her employment by giving not less than two weeks written notice to the Employer.
			\item The Employer may, in its absolute discretion, terminate this agreement and the Employee's employment at any time, for any reason with or without cause or notice.
		\end{enumerate}
	\item{\textbf{Social Media Policy}}. Employee understands that no information about his/her location, plans for the day or pictures of family members should be shared on any social media network. Employee will also not tell strangers to the family (i.e.  Employee's friends) where she is spending the day, unless authorized by one or more members of the Employer's family.  
\end{enumerate}

\subsection*{About This Agreement}
\begin{enumerate}
	\item{\textbf{Laws}}. This agreement shall be governed by the laws of the State of California.
	\item{\textbf{Entire Agreement}}. This agreement contains the entire agreement between the parties, superseding in all respects any and all prior oral or written agreements or understandings pertaining to the employment of the Employee by the Employer and shall be amended or modified only by written instrument signed by both of the parties hereto.  
	\item{\textbf{Severability}}. The parties hereto agree that in the event any article or part thereof of this agreement is held to be unenforceable or invalid then said article or part shall be struck and all remaining provisions shall remain in full force and effect. 
\end{enumerate}

\newpage
\noindent\textbf{Employer hereby agrees to be fully bound by the terms of this contract.}

\vspace{.8cm}
\noindent\tabmed{Employer Signature} \tablarge{\makebox[.5\linewidth]{\hrulefill}} \tabmed{Date\datefillin} \\*
\tabmed{} \tablarge{\textsuperscript{\mom{}}}\\* 
\noindent \tablarge{\hspace{2cm} Employer's Address and Telephone Number:} 104 N. Wilson, Apt. \#13\\
\noindent \tablarge{} Pasadena, CA 91106\\
\noindent \tablarge{} (626) 396-7081\\* \\

\noindent\textbf{Employee hereby agrees to be fully bound by the terms of this contract.}

\vspace{.8cm}
\noindent\tabmed{Employee Signature} \tablarge{\makebox[.5\linewidth]{\hrulefill}} \tabmed{Date\datefillin} \\*
\tabmed{} \tablarge{\textsuperscript{\lname{}}}

\vspace{.5cm}
\noindent Employee Address: \hrulefill

\vspace{.5cm}
\noindent Employee Telephone Number: \hrulefill

\vspace{.5cm}
\noindent Employee Email: \hrulefill
\input{notarycontent}
	\label{LastPage}
	\newpage

\section*{} %rules
\vspace{-.8cm} 
\begin{attachment}{\rules{}} \label{rules}\\

	\rowcolors{1}{lightgray}{white}
	\input{rulescontent}
\end{attachment}

\section*{} %basic
\pagestyle{empty}
\begin{attachment}{\basic{}} \label{basic}
	\input{basicdaycontent}
\end{attachment}

\section*{} %bathroom
\begin{attachment}{\bathroom{}} \label{bathroom}
	\input{bathroomcontent}
\end{attachment}

\section*{} %shopping
\newpage
\begin{attachment}{\shopping{}} \label{shopping}
	\input{shoppingcontent}
\end{attachment}

\section*{} %bigchores
\newpage
\begin{attachment}{\bigchores{}} \label{bigchores}
	\input{monthlycontent}
\end{attachment}

\section*{} %shower
\newpage
\begin{attachment}{\shower{}} \label{shower}
	\input{showercontent}
\end{attachment}

\section*{} %insulin
\newpage
\begin{attachment}{\insulin{}} \label{insulin}
	\input{insulincontent}
\end{attachment}

\section*{} %health
\newpage
\begin{attachment}{\health{}} \\ \label{health}

	\rowcolors{1}{lightgray}{white}
	\input{healthcontent}
\end{attachment}

\section*{} %notyourjob
\newpage
\begin{attachment}{\notyourjob{}} \label{notyourjob}
	something
\end{attachment}

\end{document}
