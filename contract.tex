\documentclass[]{article}
\newcommand{\mytitle}{Big Chores - One Per Week}
\newcommand{\agreementtitle}{WILSON MANOR, BEDROOM IN CONDOMINIUM APARTMENT 13, WEEK-TO-WEEK LODGING RENTAL AGREEMENT}
\overfullrule=2cm
\usepackage[margin=1in,headheight=28pt]{geometry}

\newcommand{\startdate}{the/start/date}
\newcommand{\datefillin}{\hspace{0.2cm}\makebox[3cm]{\hrulefill}}
\newcommand{\initialfillin}{\hspace{0.2cm}\makebox[1.5cm]{\hrulefill}}
\newcommand{\caregiver}{Caregiver's Name}
\newcommand{\mom}{Joan B. Harrison}
\newcommand{\rules}{Senior Care Guidelines}
\newcommand{\basic}{Basic Daily Chores}
\newcommand{\bathroom}{Bathroom Checklist}
\newcommand{\shopping}{Items To Put On Shopping List When Getting Low}
\newcommand{\big}{Big Chores, Only One To Be Done Each Week}
\newcommand{\shower}{Showers}
\newcommand{\health}{Health Records}
\newcommand{\notyourjob}{Not Your Responsibility}

\newcounter{attachmentcounter}
\renewcommand{\theattachmentcounter}{\Alph{attachmentcounter}}
\newenvironment{attachment}[1] {%
	\refstepcounter{attachmentcounter}%
	\noindent \textbf{\Large Attachment\theattachmentcounter:~{#1}}
	\noindent
}{}

\begin{document}
\title{\mytitle{}}
\author{Drafted by Colin N Keenan colinnkeenan@gmail.com}
\date{April 2015}
\maketitle
\thispagestyle{fancy}

Dear \caregiver{},

Thank you for providing this much-needed care! The following contract is to make sure we are all on the same page about responsibilities, vacation days, taxes, payments and schedules. The attached ``\rules{}'' (Attachment \ref{rules}), ``\basic{}'' (Attachment \ref{basic}), ``\bathroom{}'' (Attachment \ref{bathroom}), ``\shopping{}'' (Attachment \ref{shopping}), ``\big{}'' (Attachment \ref{big}), and ``\health{}'' (Attachment \ref{health}) offer a little more information about how our family works and how we are hoping you can assist our loved one, written down so it can be clear. While the below contract is very black and white, the attachments will be agile documents as we know things might change.

This contract, executed on \datefillin{}, between \mom{} (``Employer'') and \caregiver{} (``Employee'') has the following terms of employment:

\section{Start Date}

Employee will start employment on \textbf{\startdate{}} and continue until either party elects to terminate the relationship.

\section{Worksite Address}

Work will be performed at 104 North Wilson Ave, Apt. 13, Pasadena, CA 91106 (``Employer's Home'').

The Employee may choose to rent a room in the Employer's Home upon the terms set forth in the \agreementtitle{} (``Room Rental'').

\section{Work Schedule And Responsibilities}

The work schedule consists of sixteen hours a week done in two shifts seven days a week, and weekday medical appointments scheduled by the Employee:
\begin{enumerate}
	\item \textbf{Morning Shift}. 8am until chores are completed (typically finished at about 9am, but may finish anytime between 8:30am and 10am)
	\item \textbf{Evening Shift}. 7pm until chores are completed (typically finished at about 8pm, but may finish anytime between 7:30pm and 9pm)
	\item \textbf{16 Weekly Hours Plus Medical Appointments}. The total number of hours, not counting medical appointments, should be within half an hour of sixteen hours each week. The Employee should attempt to schedule no more than two medical appointments in the same work week. 
\end{enumerate}

You will need to check in by phone at the start of each shift and check out by phone or internet at the end of each shift as set forth in the \rules{}. Your daily morning and evening chores are set forth in \basic{}, \bathroom{}, and \shopping{}. Once a week, you will do a bigger chore from the \big{} list. You will also assist with showers a few times a month as set forth in \shower{}. You are to spend about sixteen hours each week on the chores referenced in this paragraph.

You will \textbf{not} be responsible for doing anything listed in ``\notyourjob{}'' (Attachment \ref{notyourjob}). 

\section{Notes About The Person Requiring Care}

The person you will care for, \mom{}, born February 7,1943 has been diagnosed with diabetes, scoliosis and severe arthritis, has difficulty swallowing, has had surgery for thryroid, kidney and, most recently, breast cancer, has high cholesterol and has floaters in her eyes. Though undiagnosed, she also suffers from extreme anxieties, may have auditory or visual hallucinations, and minor memory loss.

As a result of these conditions, she is unable to lift her arms above her head, walks hunched over with a pronounced limp, needs to use a four wheel walker at all times, requires help with dressing, grooming and bathing, must take care in transferring from a chair to her walker and from her walker to a toilet, may only eat certain types of foods, is very nervous about being driven in a car, and is reluctant to leave home, bathe or brush teeth.

She is able to ascend and descend stairs, use the bathroom alone and is able to feed herself.

The person you will care for can be left alone.

Caregiver's initials \initialfillin{}

\section{Compensation}

The following is the compensation you will be paid for your services.

\begin{longtable}[c]{@{}llll@{}}
\toprule
Pay Rates & Amounts & x Hours per week & Estimated Weekly
Totals\tabularnewline
Regular rate of pay & \$13.25 per hour & 12 hours &
\$159.00\tabularnewline
Medical or Dental Appointments (varies weekly, but no more than 2
appointments per week) & \$13.25 per hour & approx. 6 hours &
\$79.50\tabularnewline
+Overtime rate of pay for any hours up to and including nine hours on
the sixth (6th) and seventh (7th) workdays; for any hours worked over
nine (9) in a day\textsuperscript{*} & \$20.00 per hour & 4 hours &
\$40.00\tabularnewline
+ Overtime rate of pay of double time (2 x the regular rate of pay) for
hours worked over nine (9) hours on the sixth and seventh consecutive
day of workweek.\textsuperscript{*} & \$26.50 per hour & 0 hours &
\$0.00\tabularnewline
Weekly Totals: & & 22 hours & \$278.50\tabularnewline
\bottomrule
\end{longtable}

\textsuperscript{* As required under Wage Order no. 15-2001 ``Regulating
Wages, Hours and Working Conditions in the Household Occupations''
issued by the Industrial Welfare Commission (``Wage Order No. 15'') The
Employee's overtime rates may change as required by California law. }

\emph{Payment of Wages:}

Weekly (Every Friday). Wages will be paid directly into the Employee's
Venmo, checking or any other account as mutually agreed upon by the
Employer and the Employee by no later than 11:59 p.m. each Friday at the
end of the work week. The work week will begin on Saturday and will end
on Friday. Wages may not be paid in cash.

\textbf{{[}I've structured the work week this way because I'm having the
caregiver spend more hours on Saturday to either bathe Mom or perform
housekeeping duties. I'm presuming that our caregiver will either have a
job or other obligations on the other days. If Saturday or Sunday were
considered the 6th and 7th day of the work week, then we would have to
pay overtime for these days{]}}

\textbf{{[}I've made this provision more general, so that we don't have
to alter the contract if Venmo goes defunct or if we would be happy with
a different account. While Venmo may be convenient and we can recommend
using this service, the employee should ultimately decide to which
account her pay should go. I still need to double-check the law on
appropriate methods of payment{]} }

{[}\textbf{The live-in employees requirements apply if the Employee is a
Domestic Worker and is NOT a personal attendant.} Live in employees who
work in excess of five workdays in a workweek must be paid overtime at
the rate of one and one-half times the employees regular rate of pay for
hours worked up to and including nine hours on the sixth (6th) and
seventh (7th) workdays, and two times the employee's regular rate of pay
for all hours worked in excess of nine hours on the sixth (6th) and
seventh (7th) workdays. See,
\href{http://www.dir.ca.gov/dlse/FAQ_OvertimeExceptions.htm}{\emph{http://www.dir.ca.gov/dlse/FAQ\_OvertimeExceptions.htm}}{]}

{[}\textbf{Personal Attendant Overtime allowances DO NOT apply to our
Employee.} Wage Order 15 vaguely directed that ``personal attendant''
status only applies when ``no significant amount of work'' outside of
such caregiving duties was required. Section 1451(d) of the DWBR
specifically states ``personal attendant'' means any person employed by
a private householder or by any third-party employer recognized in the
health care industry to work in a private household, to supervise, feed,
or dress a child, or a person who by reason of advanced age, physical
disability, or mental deficiency needs supervision. The status of
personal attendant shall apply when no significant amount of work other
than the foregoing is required. For purposes of this subdivision, ``no
significant amount of work'' means work other than the foregoing did not
exceed 20 percent of the total weekly hours worked.

Thus, if more than twenty percent (20\%) of the worker's time in a week
is spent on non-caretaking duties such as general housecleaning, making
beds, cooking, laundry or other duties related to the maintenance of a
private household or the premises, the worker is NOT a Personal
Attendant. See,
\href{http://www.dir.ca.gov/dlse/DomesticWorkerBillOfRights-FAQ.html}{\emph{http://www.dir.ca.gov/dlse/DomesticWorkerBillOfRights-FAQ.html}}{]}

\emph{Payroll Log}:

The Employee will be required to submit a daily log of the hours worked
as described in the Senior Care Guidelines and Schedule.

\textbf{MILEAGE AND GENERAL EXPENSES}

Any miles driven while on the job using the employee's car will be
reimbursed at the IRS Mileage Reimbursement Rate, which covers the cost
of gasoline as well as general wear and tear on the car. Employee will
maintain a mileage log and submit to Employer for reimbursement at the
end of the pay period. The 2015 IRS mileage reimbursement rate is 57.5
cents per mile.

All other pre-approved, work-related expenses will be reimbursed at
cost. Employee will keep all receipts and submit to employer for
reimbursement at the end of each pay period.

\textbf{TAX-ADVANTAGED BENEFITS}

In addition to the wages stated above, Employer will contribute to the
following Employee expenses. These amounts are considered
``non-taxable'' compensation (up to the limits noted below), meaning
neither Employer nor Employee will pay any taxes on this portion of the
compensation. (check any that apply)

[ ] Public transportation at \$\_\_\_\_\_\_\_\_\_\_\_\_\_ per month (up to
\$130* per month)

[ ] Mobile phone service at \$\_\_\_\_\_\_\_\_\_\_\_\_ per month (up to
total amount of bill)

\textbf{{[}Tax-Advantaged Benefits Notes:} Though not required by law to
provide, I've included these tax-advantaged benefits just in case we
want to offer as a perk for a great candidate. I've omitted health
insurance, college tuition and parking as they are either too expensive
or are not needed{]}

We do have workers' compensation coverage. (circle one)

\textbf{{[}we are required to provide workers' compensation. Will double
check this requirement{]}}

\textbf{*Rates and limits vary in some locations and are subject to
change. Will call HomePay (888-273-3356) office hours M to F, 8 a.m. to
6 p.m. central time for more information if we decide to include any of
the tax-advantaged benefits}

\textbf{PAID TIME OFF}

Employee will receive the following paid time off:

\begin{itemize}
\item
  \begin{quote}
  Family Sick Leave (24 hours per year) will be accrued as set forth
  below. At least one (1) week's notice is requested for any
  appointments, etc. which may cause the Employee to miss work.
  \end{quote}
\end{itemize}

\begin{itemize}
\item
  \begin{quote}
  Vacation (\_\_ hours per year). Employee will provide any vacation
  requests at least 2 weeks in advance. (See Senior Care Guidelines for
  how this vacation will be determined)
  \end{quote}
\end{itemize}

\textbf{{[}Paid vacation time is not required under California law, but
we may want to offer as an employment perk. The pay would of course be
the typical work week hours and would not include the hours spent taking
Mom to appointments. We could offer vacation as follows or they can
start earning vacation days for a certain number of hours worked up to a
maximum per year: }

\textbf{first year: none}

\textbf{second year: one week}

\textbf{third year: two weeks}

\textbf{fourth to tenth year: three weeks }

\textbf{If we offer vacation, I will draft up more specific vacation
accrual language in the Senior Care Guidelines}

\textbf{Perhaps during those vacation dates, we can plan on having Mom
come to my house to reduce costs{]}}

\emph{Family Sick Leave Accrual}

The Employee will qualify for paid sick leave by working for the
Employer for at least 30 days within a year in California and by
satisfying a 90 day employment period. The Employee will begin to accrue
Family Sick Leave upon the first day of her employment at the rate of
one hour paid leave for every 30 hours worked. However, the Employee may
only take 24 hours of sick leave per year. Accrued paid sick leave may
be carried over to the next year, but accrued sick leave may not exceed
48 hours.

An employee can take paid leave for employee's own or a family member
for the diagnosis, care or treatment of an existing health condition or
preventive care or for specified purposes for an employee who is a
victim of domestic violence, sexual assault or stalking or as otherwise
permitted under California law. (``Family Sick Leave'').

\textbf{{[}We are required under the Healthy Workplace Healthy Family
Act of 2014 (AB 1522) to provide Family Sick Leave. Unfortunately, the
accrual requirements do not change even if you have a part time
employee. We will also have to provide sick leave rights notice. see,
http://www.dir.ca.gov/dlse/Publications/LC\_2810.5\_Notice\_(Revised-11\_2014).pdf{]}}

\textbf{HOLIDAYS}

Employer will provide the following PAID Holidays (check any that
apply):

\begin{longtable}[c]{@{}l@{}}
\toprule
PAID HOLIDAYS\tabularnewline
X\tabularnewline
\tabularnewline
\tabularnewline
\tabularnewline
X\tabularnewline
\tabularnewline
X\tabularnewline
X\tabularnewline
\bottomrule
\end{longtable}

Employer will also provide the following UNPAID holidays (check any that
apply):

\begin{longtable}[c]{@{}l@{}}
\toprule
UNPAID HOLIDAYS\tabularnewline
\tabularnewline
X\tabularnewline
X\tabularnewline
X\tabularnewline
\tabularnewline
X\tabularnewline
\tabularnewline
\tabularnewline
\bottomrule
\end{longtable}

{[}\textbf{Holiday Pay Note:}We are not required by law to provide paid
or non paid holidays, but paid major holidays might be a nice perk.
Please feel free to alter the paid/non paid holidays as see fit. If we
offer these holidays, we need to makes sure to arrange alternative
coverage, e.g. family or health care service.{]}

\textbf{TAX WITHHOLDING/REPORTING}

Employee will complete Form I-9 (available at www.uscis.gov/forms) and
provide the required documentation verifying employment eligibility
within three days of hiring.

Employer will withhold the required Social Security and Medicare taxes
from the employee's pay, along with income taxes per the Employee's
instructions on Form W-4 and all other applicable state taxes.

All tax withholdings will be remitted to the state and federal tax
agencies on or before the household employment tax deadlines. In
addition, Employer will match the employee's Social Security and
Medicare contributions and make contributions to the state and federal
unemployment insurance funds on behalf of the employee.

Employer will provide employee with Form W-2 (available at
www.irs.gov/Forms-\&-Pubs) at the end of the year (by January 31).

Employer will report employee's earnings to the Social Security
Administration so that employee receives appropriate retirement
benefits.

\textbf{Tax Withholding/Reporting Notes: Need to check against 2015
Household Employer's Guide.}
\href{http://www.edd.ca.gov/pdf_pub_ctr/de8829.pdf}{\textbf{\emph{http://www.edd.ca.gov/pdf\_pub\_ctr/de8829.pdf}}}
\textbf{I will also Call HomePay to double-check this section. For help
with the tax process, call HomePay (888-273-3356). }

\textbf{CONFIDENTIALITY}

Employee understands that any and all private information obtained about
the Employer during the course of employment, including but not limited
to medical, financial, legal and career information, is strictly
confidential and may not be disclosed to any third party for any reason.

\textbf{AT WILL EMPLOYMENT}

It is understood and agreed that:

The Employee may at any time terminate this agreement and her employment
by

giving not less than two weeks written notice to the Employer.

The Employer may, in its absolute discretion, terminate this agreement
and the Employee's employment at any time, for any reason with or
without cause or notice.

The Employee shall vacate the premises promptly upon termination.
\textbf{{[}will insert timing consistent with Rental and will
double-check whether this provision is permitted under Federal and CA
law. Also ask HomePay about this provision. {]}}

\textbf{{[}No need to insert the \$200 in checking account requirement
as the employee is ``at will'' and may be terminated at our discretion.
I have listed reasons why we might immediately terminate the employee in
the Senior Care Guidelines, but have clearly stated that we still retain
the right to terminate the employee with or without cause as courts have
held that employment manuals are to be treated as contracts between the
employee and the employer and we don't want to limit our ability to
terminate employment.{]} }

\textbf{SOCIAL MEDIA POLICY}

Employee understands that no information about his/her location, plans
for the day or pictures of family members should be shared on any social
media network. Employee will also not tell strangers to the family (i.e.
caregiver's friends) where she is spending the day, unless authorized by
one or more members of the Employer's family.

\textbf{RAISES AND REVIEWS}

After 5 years of employment, the Employee will be eligible for a raise
of \$\_\_\_or \_\_\_\%. This will be based on:

\textbf{{[}Insert what they need to do to receive a raise.{]}}

{[}\textbf{Raises and Reviews Notes:} Employers are not required to give
caregivers annual raises, but it is a common practice. Typically raises
are determined by starting with the rate of inflation (check the Bureau
of Labor Statistics website for the Consumer Price Index,
www.bls.gov/cpi/) and then add between two and five percentage points
based on performance.

Rather than starting off with the \$13.25 and keeping it at that fixed
rate for 5 years, we may want to start off with a lower wage and add
reasonable raises each year given that this is common practice for the
industry and may provide an incentive for the Employee to perform the
job well. By starting at this lower wage, we can more readily add the
companion tasks that I've suggested elsewhere.{]}

\textbf{LAWS}

This agreement shall be governed by the laws of the State of California.

\textbf{ENTIRE AGREEMENT}

This agreement contains the entire agreement between the parties,
superseding in all respects any and all prior oral or written agreements
or understandings pertaining to the employment of the Employee by the
Employer and shall be amended or modified only by written instrument
signed by both of the parties hereto.

\textbf{SEVERABILITY}

The parties hereto agree that in the event any article or part thereof
of this agreement is held to be unenforceable or invalid then said
article or part shall be struck and all remaining provisions shall
remain in full force and effect.

Employer hereby agrees to be fully bound by the terms of this contract.

Employer Signature
\_\_\_\_\_\_\_\_\_\_\_\_\_\_\_\_\_\_\_\_\_\_\_\_\_\_\_\_\_\_\_\_\_\_dated
\_\_\_\_\_\_\_\_\_\_\_

\mom{}

Employer's Address and Telephone Number:

104 N. Wilson, Apt. \#13

Pasadena, CA 91106

(626) 396-7081

Employee hereby agrees to be fully bound by the terms of this contract.

Employee Signature:
\_\_\_\_\_\_\_\_\_\_\_\_\_\_\_\_\_\_\_\_\_\_\_\_\_\_\_\_\_\_\_\_\_\_dated
\_\_\_\_\_\_\_\_\_\_

{[}Employee's Name{]}

Employee Address:
\_\_\_\_\_\_\_\_\_\_\_\_\_\_\_\_\_\_\_\_\_\_\_\_\_\_\_\_\_\_\_\_\_\_\_\_\_\_

Employee Telephone Number:
\_\_\_\_\_\_\_\_\_\_\_\_\_\_\_\_\_\_\_\_\_\_\_\_\_\_\_\_\_

Employee Email:
\_\_\_\_\_\_\_\_\_\_\_\_\_\_\_\_\_\_\_\_\_\_\_\_\_\_\_\_\_\_\_\_\_\_\_\_\_\_\_\_

\end{document}
