\documentclass[]{article}
\newcommand{\rulestitle}{Senior Care Guidelines}
\newcommand{\agreementtitle}{WILSON MANOR, BEDROOM IN CONDOMINIUM APARTMENT 13, WEEK-TO-WEEK LODGING RENTAL AGREEMENT}
\overfullrule=2cm
\usepackage[margin=1in,headheight=28pt]{geometry}
\usepackage[table]{xcolor}
\definecolor{lightgray}{gray}{0.9}
\newcommand{\checkbox}{\raisebox{2pt}{\framebox[12pt][c]{\rule[7pt]{0pt}{-.3\baselineskip}}}}

\newcommand{\startdate}{the/start/date}
\newcommand{\datefillin}{\hspace{0.2cm}\rule{3cm}{.1pt}}
\newcommand{\initialfillin}{\hspace{0.2cm}\rule{1.5cm}{.1pt}}
\newcommand{\sw}{.15\textwidth}
\newcommand{\bw}{.39\textwidth}
\newcommand{\tabtiny}[1]{\makebox[1.8em][l]{#1}\ignorespaces}
\newcommand{\tabsmall}[1]{\makebox[.14\linewidth][l]{#1}\ignorespaces}
\newcommand{\tabmed}[1]{\makebox[.22\linewidth][l]{#1}\ignorespaces}
\newcommand{\tablmed}[1]{\makebox[.33\linewidth][l]{#1}\ignorespaces}
\newcommand{\tablarge}[1]{\makebox[.55\linewidth][l]{#1}\ignorespaces}
\newcommand{\lname}{Caregiver's Name}
\newcommand{\mom}{Joan B. Harrison}
\newcommand{\allweek}{ALL-WEEK non-LIVE-IN Employee}
\newcommand{\weekend}{WEEKEND non-LIVE-IN Employee}
\newcommand{\weekday}{WEEKDAY LIVE-IN Employee}

\input{attachmenttitles}

\newcounter{attachmentcounter}
\renewcommand{\theattachmentcounter}{\Alph{attachmentcounter}}
\newenvironment{attachment}[1] {%
	\refstepcounter{attachmentcounter}%
	\noindent \textbf{\Large Attachment\theattachmentcounter:~{#1}}
	\noindent
}{}
\usepackage{multicol}
\usepackage{fancyhdr}
\pagestyle{fancy}
\fancypagestyle{empty}{
	\fancyhf{}
}

\pagestyle{empty}
\begin{document}

\textbf{Senior Care Guidelines Disclaimer: The Senior Care Documents (as defined herein) have been provided for informational and guidance purposes only. The Senior Care Documents do not contain any promises and do not constitute a contract. As an ``at will'' employee as set forth in the Senior Care Contract I may quit upon two weeks notice or be terminated by the Employer at any time for any reason with or without cause or notice. The Employer may revise the Senior Care Documents in its discretion and at any time without a written revision. However, the ``At Will Employment'' provision contained in the Senior Care Contract may not be so modified.} 

\textbf{Date:}

\section*{Family Philosophy}
We are hoping to respect Joan's wishes to remain at home where she is most comfortable so that she may continue to enjoy her retirement on her own terms. Though our mother may not have the same cognitive ability that she once had, she is not a child and should not be treated as such.  Please treat her the same way as you would like to be treated---with kindness, compassion, understanding and respect. By relying on these principles to guide your care, you will garner her trust, making your job easier and our mother happier. A request from a trusted friend is much easier to follow than a command from someone you do not like or respect. Without this trust, she may become anxious and just shut down and refuse to follow any requests. Of course, you must never use physical force or threats. Nor should you use humiliating or cruel comments as a tool for getting her to follow your instructions. These types of comments are not effective and just make her more emotionally upset.

Your responsibilities are outlined below and in the Senior Care Contract. Joan's family will continue to take care of everything else.  For example, we will continue to help Joan with any financial or legal matters, repair and maintenance of her home, health matters, as well as other day to day needs. For your convenience, we have included a list of family members and their contact information.

\section*{Family History}

Our mother is a fiercely independent person who has lived on her own for most of her life. She has been married and divorced two times. She has two children, Colin and Charlene, from the first marriage and two children, Renee and Allegra, from the second. She now has seven grandchildren and two great grandchildren. Two of her grandchildren, Rachel Ogletree and Jessica Gould, live in town, as does her older daughter, Charlene. She has developed a very close relationship with Rachel and her husband Roy as they have checked in on her and visited her often throughout these past few years. Her only brother, Bob, lives in England.

She has had many jobs throughout her life and sometimes more than one job at a time to make ends meet. She has worked as an editor at an engineering firm and a bank, a secretary and copy editor for a number of companies, as well as a stenographer. She has always been keenly interested in literature and poetry and has written scores of poems herself. She chose to retire as soon as she could to reap the benefits of all that hard work.

\section*{Ongoing Diagnosed Health Conditions}

\rowcolors{1}{white}{lightgray}
\begin{tabular}{|p{\bw}|p{\sw}|}
	\hline
	Health condition & Date noted by Kaiser Permanente under current plan (was previously on her 2nd husband's plan)\\ \hline
	History of thyroid cancer and removal of the thyroid & around 1980 (not done by Kaiser Permanente and not noted)\\ \hline
	History of Diabetes Type 2 treated with pills & late 1990's? (not noted in current Kaiser Permanente plan)\\ \hline
	History of cancer of the kidney & 1/3/2006\\ \hline
	Hypothyroidism (Low Thyroid) (as a result of removal of thyroid and not taking thyroid medication as prescribed) & 3/23/2006\\ \hline
	Essential Hypertension & 3/23/2006\\ \hline
	Scoliosis since childhood & 6/24/2006\\ \hline
	Osteoporosis (Significant Thinning Of Bone) & 1/18/2006\\ \hline
	Diabetes Type 2 With Moderate Kidney Disease (treated with insulin injections) & 08/10/2006\\ \hline
	White Coat Syndrome & 07/16/2009\\ \hline
	History Of Partial Nephrectomy (Kidney Removal) & 04/29/2010\\ \hline
	Bilateral Cataract & 05/01/2013\\ \hline
	Diabetes Type 2, Without Retinopathy & 05/01/2013\\ \hline
	Vitamin D Deficiency & 05/14/2013\\ \hline
	Abnormal Mammography & 07/15/2013\\ \hline
	Onychomycosis (Nail Fungal Infection) & 10/31/2014\\ \hline
	Onychogryphosis & 10/31/2014\\ \hline
	Dry Skin & 10/31/2014\\ \hline
	Atherosclerosis Of Aorta & 11/24/2014\\ \hline
	Diabetes Type 2 With Hyperlipidemia & 11/30/2014\\ \hline
	History Of Fall & 12/16/2014\\ \hline
	Cancer Female Breast, Right Breast & 01/21/2015\\ \hline
	Acquired Schatzkis Ring	01/21/2015\\
	\hline
\end{tabular}

\section*{Attention}

Joan may be left at home safely. She is capable of transferring from her chair to her walker, going up and down the stairs independently and is able to transfer from her walker to the toilet. 

\section*{Medication Monitoring}

\section*{Physical or Cognitive Impairments}

Joan has been diagnosed with diabetes, scoliosis and severe arthritis, has difficulty swallowing, has had surgery for thryroid, kidney and, most recently, breast cancer, has high cholesterol, has floaters in her eyes and wears eyeglasses. Though undiagnosed, Joan appears to suffer from extreme anxieties, may have auditory or visual hallucinations, minor memory loss and some hearing loss. 

As a result of these physical and cognitive conditions, she is unable to lift her arms above her head, walks hunched over with a pronounced limp, needs to use a four wheel walker at all times, requires help with dressing, grooming and bathing, must take care in transferring from a chair to her walker and from her walker to a toilet, may only eat certain types of foods, is very nervous about being driven in a car, and is reluctant to leave home, bathe, dress or brush teeth.

She will require help with the following tasks:

\begin{itemize}
\item \textbf{Bathing}. She may require help with removing her items of clothing before the shower. She is able to sit down and stand up from the bath transfer bench without help. She needs help with washing her hair and back. She may need help with drying herself off. Always make sure she wears her glasses after the shower is over.
\item \textbf{Dressing}. She may need help with undressing and dressing. Because of the extreme difficulty she has with dressing, she prefers to remain in the same clothes until the next bath. Of course, if she has a major spill that cannot be tidied up, please do help her to get changed into clean clothes.
\item \textbf{Grooming}. She will need help (or at least prompting) with brushing her hair, applying any body lotions, maintaining her fingernails and cleaning her ears. She needs help maintaining her toenails, and is currently seeing a Kaiser Permenente Dr. every three months for that. Please use your judgement to prompt or help Joan maintain her grooming.
\item \textbf{Dental}. Though she is fully capable of brushing her teeth on her own, she is very reluctant to do so. She will not use an electric toothbrush even though recommended by her Dentist; so, there's no need to get another one if the Dentist makes that suggestion again. Although it may seem she is not caring for her teeth, she nevertheless has almost all of her original teeth and no current dental issues. She has crowns and one tooth implant that has been crowned. Make yearly visits to her Dentist (or as recommended by him) and follow his advice. Her Dentist is Robin Stephen L DDS, 1245 W Huntington Dr \# 103 Arcadia, CA 91007 (626) 795-4412. Do not make appointments with other dentists that may call saying she hasn't been seen in over a year.
\item \textbf{Bathroom}. She is capable of using the downstairs bathroom on her own because the commode is at a higher level. She has difficulty sitting down and getting up from a regular toilet, but has no problems using a handicap toilet. Thus, you should only have her use a handicap toilet whenever she is in public. When she is at a family member's home, we will have that family member help her use the bathroom.
\item \textbf{Transferring from Walker to Car}. She has not required help getting into and out of the cars or buses, but takes care and must not be rushed.
\item \textbf{Stairs in Public}. Do not use stairs in public unless it is absolutely necessary. She is able to ascend and descend stairs safely at home by using the railings on either side of the stairwell. Public stairwells are often too wide or do not have any railings rendering her vulnerable to falls. If you must use a stairwell or even ascend one or two steps, you must be at her side at all times. She is not likely to attempt stairs in public, but is tempted to rest by sitting on a step. Strongly encourage her to use the seat on her walker instead because she will experience great difficulty getting up from the stairs and will refuse help.
\item \textbf{Chairs}. If the only chair available doesn't have arm rests, encourage her to sit on the seat in her walker instead because she will have diffuculty standing from the chair and will refuse help.
\end{itemize}

\section*{Typical Reactions to Receiving Care and Tips for Handling Behavior Issues}

\begin{itemize}
	\item \textbf{General Reactions}. Aside from the specific occasions outlined below, Joan is usually very accepting of offers of assistance. She will take her medications as scheduled and prompted, will obligingly eat her meals and is content to let you take care of the general housekeeping.
	\item \textbf{Refusal to Bathe}. She may refuse to bathe from time to time. Because these activities are very difficult for her, you should approach her refusal with kindness and understanding, but make it clear that she needs to take a bath to remain healthy. Of course, you must never use physical force or threats to make her bathe. Nor should you use humiliating or cruel comments as a tool for getting her to bathe. These types of comments are not effective and just make her more emotionally upset. Follow the instructions in \shower{} for when and how to prompt her to bathe.
	\item \textbf{Refusal to Drive in a Car}. She is very fearful of being driven in a car and is absolutely terrified of driving on the freeway.  You should use surface streets at all times whenever you are transporting her in a car. Again kindness, understanding and respect go a long way. Statements such as ``I know that driving can be a little scary, but do know that I will only take surface streets and I will make sure to drive extremely carefully and safely. Your comfort and wellbeing is very important to me'' could help her overcome her fears. It sometimes helps to suggest that she close her eyes while you drive. Always have her sit in the back seat. 
	\item \textbf{Requests for Sweets}. She has a very difficult time controlling her desire to eat sugary foods. Please keep any of these foods out of sight and in a safe place. She may also request that you buy sweet foods for her. Because of her diabetic condition, please follow the guidance provided below for these types of foods.
	\item \textbf{Getting Stuck}. In the past, Joan has had extreme difficulty with getting up from a chair or a couch. On those occasions, she has completely refused any offers of help and we have had to call 911 for assistance. Though she is quite stiff and does sometime complain of hip pain, it appears that she is physically capable of getting up with assistance in those situations. Only if absolutely necessary and if there is no possibility of injury to yourself, assist her to stand up.  If she refuses that assistance, please contact a family member immediately.
	\item \textbf{When to Inform the Family}. If her refusals become more prevalent or adamant, please contact a family member immediately. We are all in this together, so please feel free to call at any time with any questions you may have.
\end{itemize}

\section*{Additional Care}

list of current doctors and recurring appointments

\section*{Eating/Food Preferences or Restrictions}

\subsection*{General Instructions}. You will be responsible for providing Joan with healthy, nutritious and well balanced meals. We ask that you prepare and serve her breakfast and dinner. Lunch, as well as two healthy snacks, should be prepared in advance and they should be placed in or beside the refrigerator before you leave for the day.

She has difficulty swallowing, please try to stick to softer, easier to chew foods and cut up any meats into very small pieces.

As outlined in the Daily Schedule, you will need to measure her bood sugar levels after breakfast and dinner. She will measure these blood levels herself after lunch and at midnight before she goes to bed.

Breakfast Examples:


Lunch Examples:


Snack Examples:


Dinner Examples:


\subsection*{Ideas for Eating Out}.

\section*{Running Errands}

\section*{Visitors}
We have no specific restrictions on who is allowed to visit Joan and the
length of time they may visit. 

\subsection*{The following is a list of both in and out-of-town family and friends:}

\begin{tabular}{|p{\sw}|p{\sw}|p{\sw}|p{\sw}|}
\hline
Family Member or Friend & Relationship & Telephone Number & Email Address\\ \hline
& & &\\ \hline
\textbf{\emph{In-Town Family}} & & &\\ \hline
Charlene Gould & Daughter & home: 323-685-2055 cell: 213-948-6157 & \emph{sciencecharlene@gmail.com}\\ \hline
Rachel Ogletree & Granddaughter & cell: 626-241-2951 & bashakat@hotmail.com\\ \hline
Roy Ogletree & Grandson-in-law & & rogletree@gmail.com\\ \hline
Jessica Gould & Granddaughter & cell: 818-383-3931 & zoobze11@hotmail.com\\ \hline
Christine Bryan & Family Friend & cell: 310-956-2715 &\\ \hline
Cathy Szymoniak & Family Friend & cell: 626-7543 &\\ \hline
Buddy Szymoniak & Family Friend & &\\ \hline
& & &\\ \hline
\textbf{\emph{Out-of-Town Family}} & & &\\ \hline
Bob Giffords & Brother & &\\ \hline
Colin Keenan & Son & cell: 816-200-2658 & \emph{colinnkeenan@gmail.com}\\ \hline
Renee Fried & Daughter & cell: 240-401-3932 home: 301-983-9580 & \emph{reneefriedzone@gmail.com}\\ \hline
Michael Fried & Son-in-law & cell: 240-401-4098 home: 301-983-9580 & \emph{michael.s.fried@gmail.com}\\ \hline
Allegra Harrison & Daughter & cell: 46-780-590-3366 &\\ \hline
Matt & Son-in-law & &\\ \hline
Adrianna Gould & Granddaughter & cell: 408-515-6667 & \emph{themuppetlibrarian@gmail.com}\\ \hline
Michael Cortner & Grandson & cell: 503-939-0713 & \emph{cortnermichael@yahoo.com}\\ \hline
Renee Harrison & Granddaughter & & \emph{sunflowerenee@gmail.com}\\ \hline
Rebecca Fried & Granddaughter & cell: 240-994-3747 & RFried19@sidwell.edu\\ \hline
Daniel Fried & Grandson & cell: 240-224-4562 & \emph{dfried22@sidwell.edu}\\ \hline
Owen Lukasiewicz (Great Grandson) & Great Grandson & &\\ \hline 
Ella Lukasiewicz & Great Granddaughter & &\\ 
\hline
\end{tabular}

\subsection*{Apartment Manager}

\subsection*{Contractors such as plumbers, heating/ac, handyman and etc.}

Please contact Colin if something in the house needs to be fixed or for regular maintenance. Either you or an available family member or family friend will need to be in attendance at \textbf{all} times while the worker is in her home.

\section*{Sleep Preferences}

Joan spends a good portion of her time during the day in her easy chair.  Because she has severe arthritis and scoliosis and is very uncomfortable sleeping flat on a bed, she not only prefers to sleep in her easy chair, but her doctor recommends it. She often sleeps with the TV on. She doesn't follow a particular nap or sleep schedule, however, she will need to be woken up for her daily breakfast. Because her easy chair serves as her bed as well, we ask that you change the seat covers or sheets every few weeks as part of one of the \bigchores{} or when needed.

\section*{Communication}

You will be talking, emailing, and uploading information to Colin daily as part of your check-in and check-out procedures for each shift.

\section*{Time Off: Vacation Days and Holidays}

You are eligible for unpaid vacation days as set forth in the Senior Care Contract. Time off for holidays are treated as vacation days. Vacation days accrue as follows:

You will need to provide us with at least 2 weeks notice prior to taking any vacation days.

\section*{The Caregiver/Family Relationship}

Most communication will go through Colin since you are communicating with him several times a day as part of the chores done on every shift. You may talk to other family members as they visit or call as well, and they will play a part in overseeing your work.

\textbf{Termination Policy:}

Because you have been employed on an ``at-will'' basis, you may terminate your employment at any time upon 2 weeks written notice and we may terminate your employment for any reason with or without cause or notice as set forth in the Senior Care Contract. Though not limited to these reasons, the following may be reasons why we might choose to terminate your employment:

\begin{itemize}
\item Allowing Joan Harrison's safety to be compromised
\item Inconsistent or non-performance of agreed-upon job responsibilities
\item Re-negotiating terms of employment with Joan Harrison directly
\item Failing to report any additional monies or gifts to Colin given to the caregiver by Joan Harrison
\item Concerning issues in background checks
\item Dishonesty
\item Stealing
\item Breach of confidentiality clause in the Senior Care Contract
\item Persistent absenteeism or tardiness
\item Unapproved guests
\item Smoking while on duty or in the home
\item Alcohol consumption while on duty
\item Use of an illegal drug
\item Use of phone or computer other than as required for the job while on duty in the home (such use is fine while waiting during a medical or dental appointment)
\end{itemize}

\section*{In an Emergency}

For any life-threatening emergencies, please call 911 first. Then call the following people:

Colin Keenan (816) 200-2658

Rachel Ogletree (626) 241-2951

Charlene Gould (213) 948-6157

Colin can provide you with more detailed information about Joan Harrison's health that could be needed by the EMT team. Rachel and Charlene are locally available to help out. You should also plan on attending Joan at the hospital or other medical unit until a family member is able to join you.

For any other emergencies, please call {[}either Colin Keenan or Rachel Ogletree (or other local family member, e.g., Charlene or Chris{]} for further direction. If you are not able to reach Colin, please contact Kaiser Permanente at {[}insert phone number here{]} for further guidance. Be sure to have Joan's membership card and Health Record \textbf{{[}we should put together a detailed health record{]}} readily available at all times. Once you have either reached Colin, Rachel or Kaiser and you have been given further directions, please contact Charlene as well.

Employee Acknowledgement

I, \lname acknowledge receipt of the Senior Care Guidelines, Daily Schedule, Health Care Record and other materials reasonably related to my employment (``Senior Care Documents''). I further acknowledge and understand that I have been provided with the Senior Care Documents for informational and guidance purposes only, and that the Senior Care Documents contain no promises and do not constitute a contract. I further acknowledge and understand that I am an ``at will'' employee as set forth in the Senior Care Contract and that I may quit upon two weeks notice or be terminated by the Employer at any time for any reason with or without cause or notice.  The Employer may revise the Senior Care Documents in its discretion and at any time without a written revision. However, the ``At Will Employment'' provision contained in the Senior Care Contract may not be so modified.

\vspace{.8cm}
\noindent\tabmed{Employee Signature} \tablarge{\makebox[.5\linewidth]{\hrulefill}} \tabmed{Date\datefillin} \\*
\tabmed{} \tablarge{\textsuperscript{\lname{}}}
\end{document}
