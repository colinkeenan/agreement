\documentclass[]{article}
\overfullrule=2cm
\usepackage[margin=1in,headheight=28pt]{geometry}
\usepackage{lmodern}
\usepackage{amssymb,amsmath}
\usepackage{ifxetex,ifluatex}
\usepackage{fixltx2e} % provides \textsubscript
\ifnum 0\ifxetex 1\fi\ifluatex 1\fi=0 % if pdftex
  \usepackage[T1]{fontenc}
  \usepackage[utf8]{inputenc}
\else % if luatex or xelatex
  \ifxetex
    \usepackage{mathspec}
    \usepackage{xltxtra,xunicode}
  \else
    \usepackage{fontspec}
  \fi
  \defaultfontfeatures{Mapping=tex-text,Scale=MatchLowercase}
  \newcommand{\euro}{€}
\fi
% use upquote if available, for straight quotes in verbatim environments
\IfFileExists{upquote.sty}{\usepackage{upquote}}{}
% use microtype if available
\IfFileExists{microtype.sty}{%
\usepackage{microtype}
\UseMicrotypeSet[protrusion]{basicmath} % disable protrusion for tt fonts
}{}
\usepackage{longtable,booktabs}
\ifxetex
  \usepackage[setpagesize=false, % page size defined by xetex
              unicode=false, % unicode breaks when used with xetex
              xetex]{hyperref}
\else
  \usepackage[unicode=true]{hyperref}
\fi
\hypersetup{breaklinks=true,
            bookmarks=true,
            pdfauthor={},
            pdftitle={},
            colorlinks=true,
            citecolor=blue,
            urlcolor=blue,
            linkcolor=magenta,
            pdfborder={0 0 0}}
\urlstyle{same}  % don't use monospace font for urls
\setlength{\parindent}{0pt}
\setlength{\parskip}{6pt plus 2pt minus 1pt}
\setlength{\emergencystretch}{3em}  % prevent overfull lines
\setcounter{secnumdepth}{0}

\date{}

\begin{document}

\textbf{{[}I've kept the drafting notes in this set of guidelines for
guidance. I will delete all of those drafting notes once we have settled
on our rules. I still also need to make sure that we have not made any
promises in these guidelines.{]}}

\textbf{{[}We need to decide how we would like to divide up particular
tasks among the family. Once we've made those decisions, I will add
those family members to the document{]}}

\textbf{Senior Care Guidelines Disclaimer: The Senior Care Documents (as
defined herein) have been provided for informational and guidance
purposes only. The Senior Care Documents do not contain any promises and
do not constitute a contract. As an ``at will'' employee as set forth in
the Senior Care Contract I may quit upon two weeks notice or be
terminated by the Employer at any time for any reason with or without
cause or notice. The Employer may revise the Senior Care Documents in
its discretion and at any time without a written revision. However, the
``At Will Employment'' provision contained in the Senior Care Contract
may not be so modified.}

Note: This is a document that the family and adult or senior caregiver
will work with and develop together. The goal is for family members to
take time to fill out this document at the start of the relationship,
but update it as the loved one's needs develop and change -\/- and the
trust grows deeper between family members and the caregiver.

\textbf{Date:}

\textbf{Family Philosophy:}

Describe yourselves and how you want your parent or loved one to be
cared for. Explain what is important to you (for instance, caring and
compassionate treatment, retaining mom or dad's dignity, monitoring
medications carefully, etc.). How independent is your mom? Will she have
a large say in day-to-day needs or does she have dementia and need
direction? Describe how much involvement you will have and how much
managing you will want to do of your loved one's daily schedule.

We are hoping to respect Joan's wishes to remain at home where she is
most comfortable so that she may continue to enjoy her retirement on her
own terms. Though our mother may not have the same cognitive ability
that she once had, she is not a child and should not be treated as such.
Please treat her the same way as you would like to be treated---with
kindness, compassion, understanding and respect. By relying on these
principles to guide your care, you will garner her trust, making your
job easier and our mother happier. A request from a trusted friend is
much easier to follow than a command from someone you do not like or
respect. Without this trust, she may become anxious and just shut down
and refuse to follow any requests. Of course, you must never use
physical force or threats. Nor should you use humiliating or cruel
comments as a tool for getting her to follow your instructions. These
types of comments are not effective and just make her more emotionally
upset

Your responsibilities are outlined below and in the Senior Care
Contract. Joan's family will continue to take care of everything else.
For example, we will continue to help Joan with any financial or legal
matters, repair and maintenance of her home, health matters, as well as
other day to day needs. For your convenience, we have included a list of
family members and their contact information.

\textbf{{[}add telephone and contact list{]}}

\textbf{{[}We may want to include a very general clause of who to
contact for what issue{]}}

\textbf{Family History:}

Share a little bit about your family and the history of your loved one
so the caregiver gets to know them.

Our mother is a fiercely independent person who has lived on her own for
most of her life. She has been married and divorced two times. She has
two children, Colin and Charlene, from the first marriage and two
children, Renee and Allegra, from the second. She now has seven
grandchildren and two great grandchildren. Two of her grandchildren,
Rachel Ogletree and Jessica Gould, live in town, as does her older
daughter, Charlene. She has developed a very close relationship with
Rachel and her husband Roy as they have checked in on her and visited
her often throughout these past few years. Her only brother, Bob, lives
in England.

She has had many jobs throughout her life and sometimes more than one
job at a time to make ends meet. She has worked as an editor at an
engineering firm and a bank, a secretary and copy editor for a number of
companies, as well as a stenographer. She has always been keenly
interested in literature and poetry and has written scores of poems
herself. She chose to retire as soon as she could to reap the benefits
of all that hard work.

\textbf{Specific Diagnoses:}

Does your parent have a medical diagnosis like diabetes, congestive
heart failure or dementia? Let your caregiver know the specifics and
history here.

\textbf{{[}Colin, I'll let you fill this out{]}}

\textbf{Attention:}

Can your loved one be left alone? Some adults and seniors have certain
illnesses that require supervision at all times. If this is the case, be
very clear with your caregiver about this. What steps should your
caregiver take if another caregiver or family member is late to relieve
him or her of their duties?

Joan may be left at home safely. She is capable of transferring from her
chair to her walker, going up and down the stairs independently and is
able to transfer from her walker to the toilet. However, because she
needs to use a walker, she is unable to safely take plates or food to
and from the kitchen and will require her daily lunches and snacks to be
prepared in advance and placed next to her chair. See, Eating/Food
Preferences for guidance on serving daily meals. We also ask that you
make sure to lock the door on your way out.

\textbf{{[}please add in any other accommodations that the caregiver
will need to provide to ensure that Mom is left at home safely, e.g. any
doors that need to be left open, medications that need to be laid out,
any medic alert or monitoring systems that will need to be turned on and
etc. }

\textbf{Medication Monitoring:}

Will the caregiver need to prompt your loved one to take medication at
designated times? What happens if your mom or dad refuses to take the
medication? See the Daily Schedule for proper medication schedules and
dosages.

\textbf{{[}Colin---please fill this out{]}}

\textbf{Physical or Cognitive Impairments:}

Let your caregiver know of any physical or cognitive impairment your
loved one has. If they can't hear well, the caregiver might need to
speak louder. Do they need eye glasses? Does arthritis make getting out
of bed difficult? Will your mom or dad know how to follow the
caregiver's instructions without help? If your loved one gets confused
or anxious, let your caregiver know and share common triggers best
practices for calming mom or dad down.

Joan has been diagnosed with diabetes, scoliosis and severe arthritis,
has difficulty swallowing, has had surgery for thryroid, kidney and,
most recently, breast cancer, has high cholesterol, has floaters in her
eyes and wears eyeglasses. \textbf{{[}Does Mom have any dental
issues?{]}} Though undiagnosed, Joan appears to suffer from extreme
anxieties, may have auditory or visual hallucinations, minor memory loss
and some hearing loss. \textbf{{[}We should probably just draft up a
separate health record and include a reference here instead{]}}

As a result of these physical and cognitive conditions, she is unable to
lift her arms above her head, walks with a pronounced limp, needs to use
a four wheel walker at all times, requires help with dressing, grooming
and bathing, must take care in transferring from a chair to her walker
and from her walker to a toilet, may only eat certain types of foods, is
very nervous about being driven in a car, and is reluctant to leave
home, bathe, dress or brush teeth.

She will require help with the following tasks:

\begin{itemize}
\item
  \begin{quote}
  \textbf{Bathing once a week:} She will require help with removing her
  items of clothing before the bath. She is able to sit down and stand
  up from the bath bench without help. She needs help with washing her
  hair and body. She prefers to lay a towel across her body for
  modesty's sake. She also needs help with drying herself off. Always
  make sure she wears her glasses after the bath is over.
  \end{quote}
\end{itemize}

\begin{itemize}
\item
  \textbf{Dressing:} She requires help with undressing and dressing.
  Because of the extreme difficulty she has with dressing, she prefers
  to remain in the same clothes until the next bath. Of course, if she
  has a major spill that cannot be tidied up, please do help her to get
  changed into clean clothes.
\end{itemize}

\begin{itemize}
\item
  \begin{quote}
  G\textbf{rooming:} She needs help with brushing her hair, applying any
  body lotions, maintaining her fingernails and toenails and cleaning
  her ears. You should brush her hair every morning, apply body lotion,
  clean her ears and cut her fingernails weekly after her bath. You
  should cut her toenails monthly. She typically just lets her hair air
  dry, so you will not need to blow dry her hair.
  \end{quote}
\end{itemize}

\begin{itemize}
\item
  \textbf{Dental:} Though she is fully capable of brushing her teeth on
  her own. She is very reluctant to do so. We have provided an electric
  tooth brush, dental floss tools and a fluoride rinse to make this task
  easier for her. She should always be encouraged to brush her teeth
  every day, but never forced. \textbf{{[}we should get these items for
  her and encourage mom to use them. I think that I have an extra
  electric toothbrush. I will bring that with me tomorrow{]}}
\end{itemize}

\begin{itemize}
\item
  \begin{quote}
  \textbf{Bathroom:} She is capable of using the downstairs bathroom on
  her own because the commode is at a higher level. She has difficulty
  sitting down and getting up from a regular toilet, but has no problems
  using a handicap toilet. Thus, you should only have her use a handicap
  toilet whenever she is in public. When she is at a family member's
  home, we will have that family member help her use the bathroom.
  \end{quote}
\end{itemize}

\begin{itemize}
\item
  \begin{quote}
  \textbf{Transferring from Walker to Car:} She requires help both
  getting into and out of the car due to the fact that the seat is at a
  lower level. \textbf{{[}add more detail here. Do you help Mom get in
  and out of the car? If you do, then the caregiver will need to be
  trained to do this{]}}
  \end{quote}
\end{itemize}

\begin{itemize}
\item
  \begin{quote}
  \textbf{Stairs in Public:} Do not use stairs in public unless it is
  absolutely necessary. She is able to ascend and descend stairs safely
  at home by using the double railings on either side of the stairwell.
  Public stairwells are often too wide or do not have any railings
  rendering her vulnerable to falls. If you must use a stairwell or even
  ascend one or two steps, you must be at her side at all times.
  \textbf{{[}add more details here{]}}
  \end{quote}
\end{itemize}

\textbf{Typical Reactions to Receiving Care and Tips for Handling
Behavior Issues:}

If your loved one is very independent, make sure your caregiver knows
when and how to approach with offers of help. Does Dad reject assistance
with one activity, but accept it with another? Do you have any tips to
offer?

\textbf{{[}Colin, Charlene and Racel---try to think of any other tricky
situations that we should outline here{]}}

\begin{itemize}
\item
  \begin{quote}
  \textbf{General Reactions:} Aside from the specific occasions outlined
  below, Joan is usually very accepting of offers of assistance. She
  will take her medications as scheduled and prompted, will obligingly
  eat her meals and is content to let you take care of the general
  housekeeping.
  \end{quote}
\end{itemize}

\begin{itemize}
\item
  \textbf{Refusal to Bathe:} She may refuse to bathe from time to time.
  Because these activities are very difficult for her, you should
  approach her refusal with kindness and understanding, but make it
  clear that she needs to take a bath to remain healthy. You should
  assist her with a bath at a regular time each week to make it become a
  part of her routine. In the morning on Saturday after she has had
  breakfast is a good time because she is the most alert and in the best
  mood. An incentive such as a special treat, outing or tv show may
  help. Of course, you must never use physical force or threats to make
  her bathe. Nor should you use humiliating or cruel comments as a tool
  for getting her to bathe. These types of comments are not effective
  and just make her more emotionally upset.
\end{itemize}

\begin{itemize}
\item
  \begin{quote}
  \textbf{Refusal to Drive in a Car:} She is very fearful of being
  driven in a car and is absolutely terrified of driving on the freeway.
  You should use surface streets at all times whenever you are
  transporting her in a car. Again kindness, understanding and respect
  go a long way. Statements such as ``I know that driving can be a
  little scary, but do know that I will only take surface streets and I
  will make sure to drive extremely carefully and safely. Your comfort
  and wellbeing is very important to me'' could help her overcome her
  fears. It sometimes helps to suggest that she close her eyes while you
  drive or to have her sit in the back seat. \textbf{{[}Colin, have you
  looked into getting Mom a prescription for an anti-anxiety medication.
  Her regular doctor could do this and driving would no longer be an
  issue{]}}
  \end{quote}
\end{itemize}

\begin{itemize}
\item
  \begin{quote}
  \textbf{Requests for Sweets:} She has a very difficult time
  controlling her desire to eat sugary foods. Please keep any of these
  foods out of sight and in a safe place. She may also request that you
  buy sweet foods for her. Because of her diabetic condition, please
  follow the guidance provided below for these types of foods.
  \end{quote}
\end{itemize}

\begin{itemize}
\item
  \begin{quote}
  \textbf{Getting Stuck:} In the past, Joan has had extreme difficulty
  with getting up from a chair or a couch. On those occasions, she has
  completely refused any offers of help and we have had to call 911 for
  assistance. Though she is quite stiff and does sometime complain of
  hip pain, it appears that she is physically capable of getting up with
  assistance in those situations. Only if absolutely necessary and if
  there is no possibility of injury to yourself, assist her to stand up.
  If she refuses that assistance, please contact a family member
  immediately.
  \end{quote}
\end{itemize}

\textbf{{[}We need to make sure that the caregiver is properly trained
to give her assistance{]}}

\begin{itemize}
\item
  \begin{quote}
  \textbf{When to Inform the Family:} If her refusals become more
  prevalent or adamant, please contact a family member immediately. We
  are all in this together, so please feel free to call at any time with
  any questions you may have.
  \end{quote}
\end{itemize}

\textbf{Additional Care}:

What doctors is your loved one currently seeing or what care or
therapies (such as physical therapy) are in progress or anticipated? Do
the providers come to the house or will the caregiver need to bring your
mom or dad to appointments?

\textbf{{[}Colin, please fill this out{]}}

\textbf{Eating/Food Preferences or Restrictions:}

Discuss meals and snacks and how you want your caregiver to cook and
prepare foods throughout the day. This is a good area to instill the
importance of mom or dad getting nutritious foods they like and that are
acceptable to any dietary restrictions they may have. This is also where
you want to note any foods they cannot have because of a food allergy or
medical condition. Do they prefer foods at certain temperatures? Do they
need assistance of any kind at meal time?

\textbf{General Instructions:} You will be responsible for providing
Joan with healthy, nutritious and well balanced meals. We ask that you
prepare and serve her breakfast and dinner. Lunch, as well as two
healthy snacks, should be prepared in advance and they should be placed
in a bag in a cooler next to her chair before you leave for the day. You
will need to collect all dishes and clean up after each meal.

She has difficulty swallowing, please try to stick to softer, easier to
chew foods and cut up any meats into very small pieces.

As outlined in the Daily Schedule, you will need to measure her bood
sugar levels after breakfast and dinner. She will measure these blood
levels herself after lunch and at midnight before she goes to bed.

Breakfast Examples:

\_\_\_\_\_\_\_\_\_\_\_\_\_\_\_\_\_\_\_\_\_\_\_\_\_\_\_\_\_\_\_\_\_\_\_\_\_\_\_\_\_\_\_\_\_\_\_\_\_\_\_\_\_\_\_\_\_\_\_\_\_\_\_\_\_\_\_\_\_

\_\_\_\_\_\_\_\_\_\_\_\_\_\_\_\_\_\_\_\_\_\_\_\_\_\_\_\_\_\_\_\_\_\_\_\_\_\_\_\_\_\_\_\_\_\_\_\_\_\_\_\_\_\_\_\_\_\_\_\_\_\_\_\_\_\_\_\_\_

\_\_\_\_\_\_\_\_\_\_\_\_\_\_\_\_\_\_\_\_\_\_\_\_\_\_\_\_\_\_\_\_\_\_\_\_\_\_\_\_\_\_\_\_\_\_\_\_\_\_\_\_\_\_\_\_\_\_\_\_\_\_\_\_\_\_\_\_\_

Lunch Examples:

\_\_\_\_\_\_\_\_\_\_\_\_\_\_\_\_\_\_\_\_\_\_\_\_\_\_\_\_\_\_\_\_\_\_\_\_\_\_\_\_\_\_\_\_\_\_\_\_\_\_\_\_\_\_\_\_\_\_\_\_\_\_\_\_\_\_\_\_\_

\_\_\_\_\_\_\_\_\_\_\_\_\_\_\_\_\_\_\_\_\_\_\_\_\_\_\_\_\_\_\_\_\_\_\_\_\_\_\_\_\_\_\_\_\_\_\_\_\_\_\_\_\_\_\_\_\_\_\_\_\_\_\_\_\_\_\_\_\_

\_\_\_\_\_\_\_\_\_\_\_\_\_\_\_\_\_\_\_\_\_\_\_\_\_\_\_\_\_\_\_\_\_\_\_\_\_\_\_\_\_\_\_\_\_\_\_\_\_\_\_\_\_\_\_\_\_\_\_\_\_\_\_\_\_\_\_\_\_

Snack Examples:

\_\_\_\_\_\_\_\_\_\_\_\_\_\_\_\_\_\_\_\_\_\_\_\_\_\_\_\_\_\_\_\_\_\_\_\_\_\_\_\_\_\_\_\_\_\_\_\_\_\_\_\_\_\_\_\_\_\_\_\_\_\_\_\_\_\_\_\_\_

\_\_\_\_\_\_\_\_\_\_\_\_\_\_\_\_\_\_\_\_\_\_\_\_\_\_\_\_\_\_\_\_\_\_\_\_\_\_\_\_\_\_\_\_\_\_\_\_\_\_\_\_\_\_\_\_\_\_\_\_\_\_\_\_\_\_\_\_\_

\_\_\_\_\_\_\_\_\_\_\_\_\_\_\_\_\_\_\_\_\_\_\_\_\_\_\_\_\_\_\_\_\_\_\_\_\_\_\_\_\_\_\_\_\_\_\_\_\_\_\_\_\_\_\_\_\_\_\_\_\_\_\_\_\_\_\_\_\_

Dinner Examples:

\_\_\_\_\_\_\_\_\_\_\_\_\_\_\_\_\_\_\_\_\_\_\_\_\_\_\_\_\_\_\_\_\_\_\_\_\_\_\_\_\_\_\_\_\_\_\_\_\_\_\_\_\_\_\_\_\_\_\_\_\_\_\_\_\_\_\_\_\_

\_\_\_\_\_\_\_\_\_\_\_\_\_\_\_\_\_\_\_\_\_\_\_\_\_\_\_\_\_\_\_\_\_\_\_\_\_\_\_\_\_\_\_\_\_\_\_\_\_\_\_\_\_\_\_\_\_\_\_\_\_\_\_\_\_\_\_\_\_

\_\_\_\_\_\_\_\_\_\_\_\_\_\_\_\_\_\_\_\_\_\_\_\_\_\_\_\_\_\_\_\_\_\_\_\_\_\_\_\_\_\_\_\_\_\_\_\_\_\_\_\_\_\_\_\_\_\_\_\_\_\_\_\_\_\_\_\_\_

\textbf{Ideas for Eating Out:}

Include any places your loved one likes to eat out (even for a short
visit, like the coffee shop or an ice cream stand) and indicate how much
you will pay for a meal out. Will you also pay for the caregiver's meal?

\_\_\_\_\_\_\_\_\_\_\_\_\_\_\_\_\_\_\_\_\_\_\_\_\_\_\_\_\_\_\_\_\_\_\_\_\_\_\_\_\_\_\_\_\_\_\_\_\_\_\_\_\_\_\_\_\_\_\_\_\_\_\_\_\_\_\_\_\_

\_\_\_\_\_\_\_\_\_\_\_\_\_\_\_\_\_\_\_\_\_\_\_\_\_\_\_\_\_\_\_\_\_\_\_\_\_\_\_\_\_\_\_\_\_\_\_\_\_\_\_\_\_\_\_\_\_\_\_\_\_\_\_\_\_\_\_\_\_

\_\_\_\_\_\_\_\_\_\_\_\_\_\_\_\_\_\_\_\_\_\_\_\_\_\_\_\_\_\_\_\_\_\_\_\_\_\_\_\_\_\_\_\_\_\_\_\_\_\_\_\_\_\_\_\_\_\_\_\_\_\_\_\_\_\_\_\_\_

\textbf{Running Errands:}

Whether it's for safety or for social reasons, would you prefer that
your loved one accompany the caregiver on errands? Are there any special
considerations or equipment needed (for instance, a cane or a walker)?
Whose car will the caregiver drive?

Grocery store trips once a week. Plan on buying all needed food items
and any other necessary items for the week.

Hair cuts every two months. Please take her to get her hair cut every
two months. She prefers to go to
\_\_\_\_\_\_\_\_\_\_\_\_\_\_\_\_\_\_\_\_, phone number
\_\_\_\_\_\_\_\_\_\_\_\_\_\_\_. {[}We can omit if another family member
can take her{]}

\textbf{{[}insert any other errands that the caregiver should run{]}}

A family member will take Joan shopping for any needed clothing and any
other household items not available at the grocery store. Please keep a
list of any needed items or repairs to give to that family member.
\textbf{{[}we need to designate a particular person for this{]}}

\textbf{Ideas for Activities and Socializing:}

Discuss how engaging and creative you want the caregiver to be. Will you
be paying for any classes? Will you be supplying a weekly budget for
activities like museum trips? Does mom or dad like to go to a weekly
movie, coffee hour or book group?

Though she may be reluctant at first, we encourage you to take her out
on a weekly trip. She enjoys movies, plays or other types of shows or
trips to the local bookstore. She has expressed an interest in going to
the local swimming pool. Please contact Renee for further details if she
would like to pursue this interest. You should also accompany her on a
brief walk on a daily basis.

\textbf{{[}I know that we did not include companionship or social
activities in our contract, but these are important and we should add
time for these types of activities. If family has the time to help out
with these types of activities on a regular basis, then we could reduce
the amount of time.{]}}

Weekly Budget \$\_\_\_\_\_\_\_\_\_\_\_\_\_\_\_

\textbf{In-Home Entertainment Options:}

Does mom or dad have a favorite television show? Does he or she like to
read, be read to, listen to specific music, play cards, do puzzles,
listen to the radio or do crafts? Do they have regular visitors in the
home?

Joan enjoys watching television and is most comfortable with the
television on most of the time. At one time, she really enjoyed playing
cards and putting together puzzles. You should encourage her to engage
in these types of activities as well. We will have a deck of cards and a
couple of puzzles available for your use.

\textbf{Visitors:}

Does your loved one have regular visitors in the home? Who is
allowed/not allowed? Are there any restrictions on how long your loved
one can have visitors?

We have no specific restrictions on who is allowed to visit Joan and the
length of time they may visit. The following is a list of both in and
out-of-town family and friends:

\begin{longtable}[c]{@{}llll@{}}
\toprule
Family Member or Friend & Relationship & Telephone Number & Email
Address\tabularnewline
& & &\tabularnewline
\textbf{\emph{In-Town Family}} & & &\tabularnewline
Charlene Gould & Daughter & home: 323-685-2055

cell: 213-948-6157 &
\href{mailto:sciencecharlene@gmail.com}{\emph{sciencecharlene@gmail.com}}\tabularnewline
Rachel Ogletree & Granddaughter & cell: 626-241-2951 &
bashakat@hotmail.com\tabularnewline
Roy Ogletree & Grandson-in-law & & rogletree@gmail.com\tabularnewline
Jessica Gould & Granddaughter & cell: 818-383-3931 &
zoobze11@hotmail.com\tabularnewline
Christine Bryan & Family Friend & cell: 310-956-2715 &\tabularnewline
Cathy Szymoniak & Family Friend & cell: 626-7543 &\tabularnewline
Buddy Szymoniak & Family Friend & &\tabularnewline
& & &\tabularnewline
\textbf{\emph{Out-of-Town Family}} & & &\tabularnewline
Bob Giffords & Brother & &\tabularnewline
Colin Keenan & Son & cell: 816-200-2658 &
\href{mailto:colinnkeenan@gmail.com}{\emph{colinnkeenan@gmail.com}}\tabularnewline
Renee Fried & Daughter & cell: 240-401-3932

home: 301-983-9580 &
\href{mailto:reneefriedzone@gmail.com}{\emph{reneefriedzone@gmail.com}}\tabularnewline
Michael Fried & Son-in-law & cell: 240-401-4098

home: 301-983-9580 &
\href{mailto:michael.s.fried@gmail.com}{\emph{michael.s.fried@gmail.com}}\tabularnewline
Allegra Harrison & Daughter & cell: 46-780-590-3366 &\tabularnewline
Matt & Son-in-law & &\tabularnewline
Adrianna Gould & Granddaughter & cell: 408-515-6667 &
\href{mailto:themuppetlibrarian@gmail.com}{\emph{themuppetlibrarian@gmail.com}}\tabularnewline
Michael Cortner & Grandson & cell: 503-939-0713 &
\href{mailto:cortnermichael@yahoo.com}{\emph{cortnermichael@yahoo.com}}\tabularnewline
Renee Harrison & Granddaughter & &
\href{mailto:sunflowerenee@gmail.com}{\emph{sunflowerenee@gmail.com}}\tabularnewline
Rebecca Fried & Granddaughter & cell: 240-994-3747 &
RFried19@sidwell.edu\tabularnewline
Daniel Fried & Grandson & cell: 240-224-4562 &
\href{mailto:dfried22@sidwell.edu}{\emph{dfried22@sidwell.edu}}\tabularnewline
Owen Lukasiewicz (Great Grandson) & Great Grandson & &\tabularnewline
Ella Lukasiewicz & Great Granddaughter & &\tabularnewline
\bottomrule
\end{longtable}

\emph{Apartment Manager}

{[}fill in name and phone number{]}.

\emph{Contractors such as plumbers, heating/ac, handyman and etc.}

Please contact {[}insert responsible family member {]} if something in
the house needs to be fixed or for regular maintenance. Unless
specifically approved in advance by {[}insert responsible family member
here{]}, either you or an available family member or family friend will
need to be in attendance at \textbf{all} times while the worker is in
her home.

\textbf{Sleep Preferences:}

Talk about your loved one's typical sleep patterns and needs. Explain
any particular rituals or habits your mom or dad likes to follow. Do
they need a nap after lunch or following a doctor's appointment? Should
the room be dark? Do they want the temperature warm or cool?

Joan spends a good portion of her time during the day in her easy chair.
Because she has severe arthritis and scoliosis and is very uncomfortable
sleeping flat on a bed, she not only prefers to sleep in her easy chair,
but her doctor recommends it. She often sleeps with the TV on. She
doesn't follow a particular nap or sleep schedule, however, she will
need to be woken up for her daily breakfast. Because her easy chair
serves as her bed as well, we ask that you change the seat covers or
sheets on a weekly basis and make sure that her pillow and blanket are
readily available on the side table next to her chair (or some other
type of shelf)

\textbf{Communication:}

Would you like to hear from the caregiver throughout the day or get
overall daily or weekly updates? What particulars do you want to know
about immediately? What can wait? Do you want a phone call, text or
email? Do you want a written record of the day? How would you like to
discuss concerns that arise? Is certain communication required with
other paid caregivers? Is the caregiver allowed to discuss your loved
one's care with other relatives? Which ones?

\textbf{{[}Colin---perhaps you can offer suggestions for this
section{]}}

\textbf{Time Off:}

Describe how you want your caregiver to request vacation or sick days
(email or verbal request). How much notice does she need to give you?

\emph{Vacation Days and Holidays}: You are eligible for paid vacation
days, as well as paid and unpaid holidays as set forth in the Senior
Care Contract. Vacation days accrue as follows:

\textbf{{[}I will draft up specific accrual language here based on what
we decide{]}}

You will need to provide us with at least 2 weeks notice prior to taking
any vacation days.

\textbf{The Caregiver/Family Relationship:}

When will you conduct performance reviews (90-days, annually, etc.)?
Will you have regular updates and meetings about how the job is going?
How do you want your caregiver to feel within your family? What role do
you see her playing?

\_\_\_\_\_\_\_\_\_\_\_\_\_\_\_\_\_\_\_\_\_\_\_\_\_\_\_\_\_\_\_\_\_\_\_\_\_\_\_\_\_\_\_\_\_\_\_\_\_\_\_\_\_\_\_\_\_\_\_\_\_\_\_\_\_\_\_\_\_\_\_\_\_\_\_\_

\_\_\_\_\_\_\_\_\_\_\_\_\_\_\_\_\_\_\_\_\_\_\_\_\_\_\_\_\_\_\_\_\_\_\_\_\_\_\_\_\_\_\_\_\_\_\_\_\_\_\_\_\_\_\_\_\_\_\_\_\_\_\_\_\_\_\_\_\_\_\_\_\_\_\_\_

\_\_\_\_\_\_\_\_\_\_\_\_\_\_\_\_\_\_\_\_\_\_\_\_\_\_\_\_\_\_\_\_\_\_\_\_\_\_\_\_\_\_\_\_\_\_\_\_\_\_\_\_\_\_\_\_\_\_\_\_\_\_\_\_\_\_\_\_\_\_\_\_\_\_\_

\_\_\_\_\_\_\_\_\_\_\_\_\_\_\_\_\_\_\_\_\_\_\_\_\_\_\_\_\_\_\_\_\_\_\_\_\_\_\_\_\_\_\_\_\_\_\_\_\_\_\_\_\_\_\_\_\_\_\_\_\_\_\_\_\_\_\_\_\_\_\_\_\_\_\_\_

\textbf{{[}We need to figure out exactly how we will monitor the
caregiver's work. It should be a combination of regular visits by family
and friends coupled with communications to Colin, Me and Allegra. I will
see Cathy and Christine this week and will ask if they might be able to
stop by from time to time. Perhaps, we can have Christine come and do
the deep cleaning once a month and ask Cathy to stop by wPerhaps we can
discuss all of this once I arrive in town{]}}

\textbf{Termination Policy:}

Because you have been employed on an ``at-will'' basis, you may
terminate your employment at any time upon 2 weeks written notice and we
may terminate your employment for any reason with or without cause or
notice as set forth in the Senior Care Contract. Though not limited to
these reasons, the following may be reasons why we might choose to
terminate your employment:

\begin{itemize}
\item
  \begin{quote}
  Allowing Joan Harrison's safety to be compromised
  \end{quote}
\end{itemize}

\begin{itemize}
\item
  \begin{quote}
  Inconsistent or non-performance of agreed-upon job responsibilities
  \end{quote}
\end{itemize}

\begin{itemize}
\item
  \begin{quote}
  Re-negotiating terms of employment with Joan Harrison directly
  \end{quote}
\end{itemize}

\begin{itemize}
\item
  \begin{quote}
  Failing to report any additional monies or gifts to {[}insert
  responsible family member here{]} given to the caregiver by Joan
  Harrison
  \end{quote}
\end{itemize}

\begin{itemize}
\item
  \begin{quote}
  Concerning issues in background checks
  \end{quote}
\end{itemize}

\begin{itemize}
\item
  \begin{quote}
  Dishonesty
  \end{quote}
\end{itemize}

\begin{itemize}
\item
  \begin{quote}
  Stealing
  \end{quote}
\end{itemize}

\begin{itemize}
\item
  \begin{quote}
  Breach of confidentiality clause in the Senior Care Contract
  \end{quote}
\end{itemize}

\begin{itemize}
\item
  \begin{quote}
  Persistent absenteeism or tardiness
  \end{quote}
\end{itemize}

\begin{itemize}
\item
  \begin{quote}
  Unapproved guests
  \end{quote}
\end{itemize}

\begin{itemize}
\item
  \begin{quote}
  Smoking while on duty or in the home
  \end{quote}
\end{itemize}

\begin{itemize}
\item
  \begin{quote}
  Alcohol consumption while on duty
  \end{quote}
\end{itemize}

\begin{itemize}
\item
  \begin{quote}
  Use of an illegal drug
  \end{quote}
\end{itemize}

\begin{itemize}
\item
  \begin{quote}
  Overuse of cell phone or computer while on duty
  \end{quote}
\end{itemize}

\begin{itemize}
\item
  \begin{quote}
  If living in the Employer's Home, Delinquent Rent as set forth in the
  Room Lease \textbf{{[}will link this with terms in the lease{]}}
  \end{quote}
\end{itemize}

\textbf{In an Emergency:}

What do you want your caregiver to do in an emergency? After calling
emergency services, who else should be notified? List names and numbers
here or share this emergency checklist.

For any life-threatening emergencies, please call 911 first. Then call
the following people:

Colin Keenan (816) 200-2658

Rachel Ogletree (626) 241-2951

Charlene Gould (213) 948-6157

Colin can provide you with more detailed information about Joan
Harrison's health that could be needed by the EMT team. Rachel and
Charlene are locally available to help out. You should also plan on
attending Joan at the hospital or other medical unit until a family
member is able to join you.

For any other emergencies, please call {[}either Colin Keenan or Rachel
Ogletree (or other local family member, e.g., Charlene or Chris{]} for
further direction. If you are not able to reach Colin, please contact
Kaiser Permanente at {[}insert phone number here{]} for further
guidance. Be sure to have Joan's membership card and Health Record
\textbf{{[}we should put together a detailed health record{]}} readily
available at all times. Once you have either reached Colin, Rachel or
Kaiser and you have been given further directions, please contact
Charlene as well.

\textbf{{[}I took out the signature lines originally contained in this
document. We want the guidelines to be a document that we may revise at
any time, and NOT a legally binding contract. {]}\\}

\textbf{{[}I will draft up an acknowledgement of receipt of Senior Care
Rules, Health Record and any other materials. It is very important that
the employee signs this as we DO NOT want the employee manual to be
rendered an enforceable contract. That could eviscerate our at-will
employment clause{]} }

Employee Acknowledgement

I, \_\_\_\_\_\_\_\_\_\_\_\_\_\_\_\_\_\_\_\_\_\_ acknowledge receipt of
the Senior Care Guidelines, Daily Schedule, Health Care Record and other
materials reasonably related to my employment (``Senior Care
Documents''). I further acknowledge and understand that I have been
provided with the Senior Care Documents for informational and guidance
purposes only, and that the Senior Care Documents contain no promises
and do not constitute a contract. I further acknowledge and understand
that I am an ``at will'' employee as set forth in the Senior Care
Contract and that I may quit upon two weeks notice or be terminated by
the Employer at any time for any reason with or without cause or notice.
The Employer may revise the Senior Care Documents in its discretion and
at any time without a written revision. However, the ``At Will
Employment'' provision contained in the Senior Care Contract may not be
so modified.

Employee Signature:
\_\_\_\_\_\_\_\_\_\_\_\_\_\_\_\_\_\_\_\_\_\_\_\_\_\_\_\_\_\_\_\_\_\_\_\_\_

Printed Name:
\_\_\_\_\_\_\_\_\_\_\_\_\_\_\_\_\_\_\_\_\_\_\_\_\_\_\_\_\_\_\_\_\_\_\_\_\_\_\_\_\_\_

Date: \_\_\_\_\_\_\_\_\_\_\_\_\_\_\_

\end{document}
